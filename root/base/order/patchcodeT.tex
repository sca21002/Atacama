%%%%%%%%%%%%%%%%%%%%%%%%%%%%%%%%%%%%%%%%%%%%%%%%%%%%%%%%%%%%%%%%%%%%%%%%%%%%%%
% 
% patchcodeT.tex: provide a pachcode T background image with pure LaTeX
% 
% Usage: \include this file in the preamble of your document
% Author: Helge Knüttel, April 2011
% 
%%%%%%%%%%%%%%%%%%%%%%%%%%%%%%%%%%%%%%%%%%%%%%%%%%%%%%%%%%%%%%%%%%%%%%%%%%%%%%

\usepackage{eso-pic}    % background pictures
\usepackage{calc}        % Rechnen mit Längen
\usepackage{picture}

\usepackage[dvips,pdftex,top=32mm,left=32mm]{geometry}

% Define image dimensions
\newlength{\picturewidth}
\setlength{\picturewidth}{\paperwidth - 2cm}
\newlength{\pictureheight}
\setlength{\pictureheight}{\paperheight - 2cm}

% Define size and position of textarea
\setlength{\textwidth}{\paperwidth - 2cm - 0.9in - 0.9in}
\setlength{\textheight}{\paperheight - 2cm - 0.9in - 0.9in}
%\setlength{\top}{1cm + 0.9in}
%\setlength{\left}{1cm + 0.9in}


% bar dimensions
\newlength{\narrowbarwidth}
\newlength{\broadbarwidth}
\newlength{\spacerbarwidth}
\newlength{\verticalbarlength}
\newlength{\horizontalbarlength}
% Where we start to draw a line
\newlength{\barxpos}
\newlength{\barypos}

%%%%%%%%%%%%%%%%%%%%%%%%%%%%%%%%%
% patchcode T pattern
%%%%%%%%%%%%%%%%%%%%%%%%%%%%%%%%%
\newcommand{\patchcodeT}{
  % patchcode T: Order of bars starting from edge of paper: narrow, broad, narrow, broad

\begin{picture}(\picturewidth,\pictureheight)

    % Define bar dimensions
    \setlength{\narrowbarwidth}{0.08in}
    \setlength{\broadbarwidth}{0.20in}
    \setlength{\spacerbarwidth}{0.08in}

    \setlength{\verticalbarlength}{\pictureheight}
    \setlength{\horizontalbarlength}{\picturewidth - 1.8in}


    %%%%%%%%%%%%%%%%%%%%
    % left vertical bars
    %%%%%%%%%%%%%%%%%%%%
    \setlength{\barypos}{0in}
    % Apparently, lines are centered around the given position. Therefore, add \linethickness\2 to \barxpos

    \linethickness{\narrowbarwidth}
    \setlength{\barxpos}{%
      \narrowbarwidth/2
    }
    \put(\barxpos,\barypos){\line(0,1){\verticalbarlength}}

    \linethickness{\broadbarwidth}
    \setlength{\barxpos}{%
      \narrowbarwidth + \spacerbarwidth + \broadbarwidth/2
    }
    \put(\barxpos,\barypos){\line(0,1){\verticalbarlength}}

    \linethickness{\narrowbarwidth}
    \setlength{\barxpos}{%
      \narrowbarwidth + \spacerbarwidth + \broadbarwidth + \spacerbarwidth + \narrowbarwidth/2
    }
    \put(\barxpos,\barypos){\line(0,1){\verticalbarlength}}

    \linethickness{\broadbarwidth}
    \setlength{\barxpos}{%
      \narrowbarwidth + \spacerbarwidth + \broadbarwidth + \spacerbarwidth + \narrowbarwidth + \spacerbarwidth + \broadbarwidth/2
    }
    \put(\barxpos,\barypos){\line(0,1){\verticalbarlength}}


    %%%%%%%%%%%%%%%%%%%%%
    % right vertical bars
    %%%%%%%%%%%%%%%%%%%%%
    \setlength{\barypos}{0in}
    % Apparently, lines are centered around the given position. Therefore, add \linethickness\2 to \barxpos

    \linethickness{\narrowbarwidth}
    \setlength{\barxpos}{%
      \picturewidth - ( \narrowbarwidth/2 )
    }
    \put(\barxpos,\barypos){\line(0,1){\verticalbarlength}}

    \linethickness{\broadbarwidth}
    \setlength{\barxpos}{%
      \picturewidth - ( \narrowbarwidth + \spacerbarwidth + \broadbarwidth/2 )
    }
    \put(\barxpos,\barypos){\line(0,1){\verticalbarlength}}

    \linethickness{\narrowbarwidth}
    \setlength{\barxpos}{%
      \picturewidth - ( \narrowbarwidth + \spacerbarwidth + \broadbarwidth + \spacerbarwidth + \narrowbarwidth/2 )
    }
    \put(\barxpos,\barypos){\line(0,1){\verticalbarlength}}

    \linethickness{\broadbarwidth}
    \setlength{\barxpos}{%
      \picturewidth - ( \narrowbarwidth + \spacerbarwidth + \broadbarwidth + \spacerbarwidth + \narrowbarwidth + \spacerbarwidth + \broadbarwidth/2 )
    }
    \put(\barxpos,\barypos){\line(0,1){\verticalbarlength}}

    %%%%%%%%%%%%%%%%%%%%%%%
    % upper horizontal bars
    %%%%%%%%%%%%%%%%%%%%%%%
    \setlength{\barxpos}{.9in}
    % Apparently, lines are centered around the given position. Therefore, add \linethickness\2 to \barypos

    \linethickness{\narrowbarwidth}
    \setlength{\barypos}{%
      \pictureheight - ( \narrowbarwidth/2 )
    }
    \put(\barxpos,\barypos){\line(1,0){\horizontalbarlength}}

    \linethickness{\broadbarwidth}
    \setlength{\barypos}{%
      \pictureheight - ( \narrowbarwidth + \spacerbarwidth + \broadbarwidth/2 )
    }
    \put(\barxpos,\barypos){\line(1,0){\horizontalbarlength}}

    \linethickness{\narrowbarwidth}
    \setlength{\barypos}{%
      \pictureheight - ( \narrowbarwidth + \spacerbarwidth + \broadbarwidth + \spacerbarwidth + \narrowbarwidth/2 )
    }
    \put(\barxpos,\barypos){\line(1,0){\horizontalbarlength}}

    \linethickness{\broadbarwidth}
    \setlength{\barypos}{%
      \pictureheight - ( \narrowbarwidth + \spacerbarwidth + \broadbarwidth + \spacerbarwidth + \narrowbarwidth + \spacerbarwidth + \broadbarwidth/2 )
    }
    \put(\barxpos,\barypos){\line(1,0){\horizontalbarlength}}

    %%%%%%%%%%%%%%%%%%%%%%%
    % lower horizontal bars
    %%%%%%%%%%%%%%%%%%%%%%%
    \setlength{\barxpos}{.9in}
    % Apparently, lines are centered around the given position. Therefore, add \linethickness\2 to \barypos

    \linethickness{\narrowbarwidth}
    \setlength{\barypos}{%
      \narrowbarwidth/2
    }
    \put(\barxpos,\barypos){\line(1,0){\horizontalbarlength}}

    \linethickness{\broadbarwidth}
    \setlength{\barypos}{%
      \narrowbarwidth + \spacerbarwidth + \broadbarwidth/2
    }
    \put(\barxpos,\barypos){\line(1,0){\horizontalbarlength}}

    \linethickness{\narrowbarwidth}
    \setlength{\barypos}{%
      \narrowbarwidth + \spacerbarwidth + \broadbarwidth + \spacerbarwidth + \narrowbarwidth/2
    }
    \put(\barxpos,\barypos){\line(1,0){\horizontalbarlength}}

    \linethickness{\broadbarwidth}
    \setlength{\barypos}{%
      \narrowbarwidth + \spacerbarwidth + \broadbarwidth + \spacerbarwidth + \narrowbarwidth + \spacerbarwidth + \broadbarwidth/2
    }
    \put(\barxpos,\barypos){\line(1,0){\horizontalbarlength}}

\end{picture}
} % end of patchcodeT



% A picture centered on the page background
\newcommand\BackgroundPicture{%
   \put(0,0){%
     \parbox[b][\paperheight]{\paperwidth}{%
       \vfill
       \centering
       \patchcodeT
       \vfill
    }
  }
}  

% Add background picture to all pages
\AddToShipoutPicture{\BackgroundPicture}

